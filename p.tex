%https://github.com/edasubert/beamerMaterialDesign


\documentclass{beamer}
\usefonttheme{professionalfonts}
\usepackage{fontspec}
\usepackage{mathtools}
\usepackage{unicode-math}
\usepackage{pgfpages}
\usepackage{comment}

\setbeameroption{show notes on second screen}

\title{Eventually Consistent Transactions}
\date{\today}
\author{Jan Bajer\\Adam Pioterek}

\usetheme{material}

\useLightTheme{}
\usePrimaryIndigo{}
\useAccent{4caf50}{087f23}


\begin{document}

\begin{frame}
\titlepage{}
\note{Hello!}
\end{frame}

\section{Jan}

\begin{frame}{Opis problemów wg wstępu z Cloud Types}
\end{frame}

\begin{frame}{Definicje Sequential, Eventual, Revision Consistency}
\end{frame}

\begin{frame}{Revision Graphs, porównanie rewizji i transakcji}
\end{frame}

\begin{frame}{Historie?}
\end{frame}

\section{Adam}

\begin{frame}{Cloud Types}
\end{frame}

\begin{frame}{Przedstawienie app. z 2. art., kod}
\end{frame}

\begin{frame}{Yield w rev. diagrams}
\end{frame}

\begin{frame}{Storage Consistency z 2.5}
\end{frame}

\begin{comment}

\section{Intro}

\begin{frame}{Overlay}
    \begin{card}
        \begin{itemize}[<+->]
            \item hello
            \item there
        \end{itemize}
    \end{card}
    \note{This shows overlay}
\end{frame}

\begin{frame}{Card}
    \begin{card}
        {\color{accent} \textbackslash{}begin\{card\}\\[2mm]}
        \null\qquad \textit{[your content here]}\\[2mm]
        {\color{accent} \textbackslash{}end\{card\}}
    \end{card}
\end{frame}

\section{Unicode}

\begin{frame}{TitleCard}
    \begin{card}[Some unicode in formulas:]
        \begin{equation}
            α² + β² = γ²
        \end{equation}
        And a fraction: ½.
    \end{card}
    \note{Wow! Unicode works}
\end{frame}

\end{comment}

\end{document}
